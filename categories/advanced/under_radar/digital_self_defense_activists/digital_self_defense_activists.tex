\documentclass[11pt]{beamer}
\usetheme{PaloAlto}
\usepackage[utf8]{inputenc}
\usepackage[english]{babel}
\usepackage{amsmath}
\usepackage{amsfonts}
\usepackage{amssymb}
\author{PufferFish}
\title{Digital Self Defense for Activist}
%\setbeamercovered{transparent} 
%\setbeamertemplate{navigation symbols}{} 
%\logo{} 
%\institute{} 
%\date{} 
%\subject{} 
\begin{document}

\begin{frame}
\titlepage
\end{frame}

\begin{frame}
\tableofcontents
\end{frame}
\section{Thread Modeling}
\begin{frame}{Basics}
Security is not about using tools, is about understanding the thread and how to counter it.
\end{frame}
\subsection{The five questions}
\begin{frame}{The five questions}
\begin{enumerate}
\item What do you want to protect?
\item Who do you want to protect it from?
\item How  likely is that you will have to protect it?
\item How bad are the consequences of failing?
\item How many effort do you want to spend to protect it?
\end{enumerate}
\end{frame}
\begin{frame}{What do you want to protect?}
List the  data  you want to  protect
\end{frame}
\begin{frame}{Who do you want to protect it from?}
Who may want this data (government, companies, individuals...)
What is it going to do with your data
\end{frame}
\begin{frame}{How  likely is that you will have to protect it?}
Who may want this data (government, companies, individuals...)
What is it going to do with your data
\end{frame}
\begin{frame}{Threat vs Risk}
Threat: a bad thing that can happen
Risk: how likely is that to happen

The thread is the diying from cancer, the risk is greater if you smoke.
\end{frame}
\section{Security basic tools}
\begin{frame}{Browsing}
HTTPS Everywhere
Privacy badger
Firefox
Duck Duck Go 
\end{frame}
\begin{frame}{Mobile phone}
Set a lock code or password
This code won't prevent the data leak if the person has a prolonged (or not really...) psychical access to  the device
\end{frame}
\begin{frame}{Mobile device vs Desktop}
The most secure data should never be used on a mobile device
\end{frame}
\begin{frame}{The insecure machine/Burning phone}
A machine that won't have any personal information
Will never be turned on on your workplace, home  or  even city/country
You are ready to throw away

For more details consult\ref{https://crimethinc.com/2017/03/27/burner-phone-best-practices}{CrimethInc - Burner phone Best Practices}
\end{frame}
\begin{frame}{Mobile device vs Desktop}

\end{frame}
\section{Encryption}
Full disk encryption
\subsection{Back-Up}
\begin{frame}{Mobile phone}
Depending on your thread model your attacker may success just by deleting your data, without unencrypting it.
Create a copy and store it somewhere else:
If you the data is really precious you can:
\begin{itemize}
\item Send an encrypted copy to someone and send them the key (by another communication method)
\item Split the key in several chunks and send them to multiple  persons
\item Be imaginative
\end{itemize}
\end{frame}
\subsection{Passwords}
\begin{frame}{What is a secure password and what is not}

\end{frame}
+15 characters
\begin{frame}{How to generate passwords}

\end{frame}
\begin{frame}{How to generate passwords}

\end{frame}
\begin{frame}{How to store passwords}

\end{frame}
\section{Going to a protest}
\subsection{Your devices WILL IDENTIFY YOU (and your friends)}
\begin{frame}{Just for being there}
The calls you do and the sites you visit
The GPS location/even if  the GPS  is off
\end{frame}
\begin{frame}{If they get you}
Browsing history
Passwords
Media content
Contacts
Social media accounts
You can be olbigated to unlock/deccrypt your device
\end{frame}
\subsection{Hide your identity}
\begin{frame}{Wear a mask}
Your face is the easiest way to recognise you
\end{frame}
\begin{frame}{But not the only one...}
\begin{itemize}
\item Clothing
\item Height
\item Distance between your shoulders/ears/nose/eyes...
\item The people surrounding you
\item Even gestures!
\end{itemize}
\end{frame}
\section{Communication}
\begin{frame}{Communication}


\end{frame}
\subsection{End-to-end encryption}
\begin{frame}{What is encryption}
Taking plaintext and a random generated key and performing mathematical operations to hide it's original content.
Decryption is taking the ciphered test, and WITH THE CORRECT KEY performing mathematical operations to recover the original plaintext.
\end{frame}
\begin{frame}{What does it protect me from?}
Someone that wants to look at the data

\end{frame}
\begin{frame}{What does it NOT protect me from? Limits of the encryption}
If the device is already compromised (Key logger, surveiled...  -  Microsoft). 
Anyone surveying the network (live or just stocking the information) will still know that I talked with you today at 22h for  35 minutes
They will know that after I called  my doctor
They will know that then I visited a pharmacy web site
\end{frame}
\begin{frame}{How does it work?}
Only the persons taking part on the conversation can decrypt the message, not the service providers.
\begin{itemize}
\item Marie and Robert want to hide the content of their conversations
\item Marie and Robert will generate a secure key 
\end{itemize}

\end{frame}
\subsection{Signal}
\begin{frame}{Why?}
End to end encryption
Created by a non-profit organization
Open source
\end{frame}
\begin{frame}{What does it do?}
Messages and calls
Encrypt messages BETWEEN SIGNAL USERS
Allows group conversations
Has a desktop version
\end{frame}
\begin{frame}{It also...}
Send disappearing messages
Disable lock-screen notifications
\end{frame}
\begin{frame}{Safety numbers...}
It allows you to verify that the conversation hasn't been intercepted by someone
\end{frame}
\begin{frame}{What does it doesn't do?}
Encrypt old SMS communications
Encrypt  messages on the phone, this requires disk  encryption
IT DOES: Store minimum meta-data
\end{frame}
\begin{frame}{}
It's useless if the device is already compromised
\end{frame}
\begin{frame}{Signal vs WhatsApp}
Depending on your risk analysis:\\
If you want to avoid being detected Whatsapp is more widely used.
\end{frame}
\section{Anonimity}
\subsection{TOR}
\begin{frame}{What is TOR}
Software and a network
It allows you to hide your IP address, your location
\end{frame}
\begin{frame}{How does it work}
The TOR network is composed by servers (nodes)
The request traverses three nodes before reaching the final destination
It's encrypted between each node
\end{frame}
\begin{frame}{Limits}
The global adversary
\end{frame}

https://freedom.press/training/preventative-mobile-security-tips-activists/

https://tacticaltech.org/

https://ssd.eff.org/

securityinabox.org/en/

https://ssd.eff.org/en/playlist/activist-or-protester#attending-protests-international

guardian project informa  cam

https://www.frontlinedefenders.org/en/digital-security-resources

 Malte Spitz - http://www.zeit.de/datenschutz/malte-spitz-data-retention
\end{document}