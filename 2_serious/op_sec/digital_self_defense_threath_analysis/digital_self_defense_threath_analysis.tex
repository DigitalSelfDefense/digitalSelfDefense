\documentclass[11pt]{beamer}
\usetheme{PaloAlto}
\usepackage[utf8]{inputenc}
\usepackage[english]{babel}
\usepackage{amsfonts}
\usepackage{amssymb}
\author{PufferFish}
\title{Digital Self Defense for Activist}

\begin{document}
\begin{frame}
\titlepage
\end{frame}

\begin{frame}
\tableofcontents
\end{frame}

\section{Threat Modeling}
\begin{frame}{Basics}
Security is not about using tools, is about understanding the thread and how to counter it.
\end{frame}
\begin{frame}{Digital threat}
We have to \textbf{learn} how to recognize  digital threat
\end{frame}
\begin{frame}{Personal threats vs group threats}
Groups have specific threats
\end{frame}
\subsection{Common threats}
\begin{frame}{What does information mean? - WOPS}
\begin{itemize}
\item Work related: documents, pictures, videos, databases...
\item Operational information: SMS, phone numbers, meeting minutes...
\item Personal information: anything that identifies us as individuals and as participants on a collective
\item Smart data: meta-data generated from our devices, our communications, internet search, social networks...
\end{itemize}
\end{frame}
\begin{frame}{Common threats}
\begin{itemize}
\item Theft
\item Lost
\item Inspection
\item Destruction
\item Sharing/Disclosure
\item Surveillance/Monitoring
\item Targeted malware
\end{itemize}
\end{frame}
\subsection{The five questions}
\begin{frame}{The five questions}
\begin{enumerate}
\item What do you want to protect?
\item Who do you want to protect it from?
\item How  likely is that you will have to protect it?
\item How bad are the consequences of failing?
\item How many effort do you want to spend to protect it?
\end{enumerate}
\end{frame}
\begin{frame}{What do you want to protect?}
List the  data  you want to  protect
\end{frame}
\begin{frame}{Who do you want to protect it from?}
Who may want this data (government, companies, individuals...)\\
What is it going to do with your data
\end{frame}
\begin{frame}{How likely is that you will have to protect it?}
How much effort will invest for getting it?
\end{frame}
\begin{frame}{Threat vs Risk}
Threat: a bad thing that can happen\\
Risk: how likely is that to happen\\
The threat is dying from cancer, the risk is greater if you smoke.
\end{frame}
\subsection{Exercise: Working with actors}
\begin{frame}{Actor identification - Description}
Every day  we interact with many  person\\
Visualizing these actors will help making us aware  of their presence\\
We have to know them  so we  can set an appropriate acceptance, deterrence or protection strategy.\\
\end{frame}
\begin{frame}{Actor identification - Exercise description}
\begin{enumerate}
\item Create small groups
\item Brainstorm all the actors you interact with
\item Divide the actors on three groups: Direct, Indirect, Potential
\end{enumerate}
\end{frame}
\begin{frame}{Actor identification - Exercise notes}
What is an actor?
\begin{itemize}
\item \emph{An actor can be any individual, group of individuals, organization, company...}
\end{itemize}
Think also about the hidden actors you interact with
\begin{itemize}
\item\emph{City council, ISP, app/device makers...}
\end{itemize}
\end{frame}
\begin{frame}{Actor identification - Exercise notes II}
World is big
\begin{itemize}
\item\emph{Think about local, regional, national and international  actors when needed.}\end{itemize}
An interaction:
\begin{itemize}
\item\emph{Working together, sharing information, opposition, hangout together...}
\end{itemize}
\end{frame}
\begin{frame}{Actor mapping - Description}
Actors interact with each other
\end{frame}
\begin{frame}{Actor mapping - Exercise description}
\begin{enumerate}
\item Place the actors on the board with you on the center
\item Map the relation between the different actors following the legend
\item Identify the weight of each actor
\end{enumerate}
Tip: You can make short statements about the relations
\end{frame}
\begin{frame}{Actor mapping - Relation legend}
\begin{itemize}
\item Close: Positive and regular relation (straight line)
\item Alliances: Coordinate their activities (double line)
\item Conflict: Antagonistic relationship (joggled  line)
\item Violent conflict: Aggression between both parties(double joggled line)
\item Compulsion: The actor A has influence over the actor B (an arrow)
\item Interdependent: they are dependent  (double arrow)
\item Weak or unknown: Hard to identify
\end{itemize}
\end{frame}
\section{Digital threats}
\begin{frame}{How can I be attacked?}
\begin{itemize}
\item Phishing
\item Unsecured wireless
\item Removable media
\item Mobile devices
\item Malware
\item Spyware
\end{itemize}
\end{frame}
\begin{frame}{Spyware}
Malicious software that tracks your activity
\begin{itemize}
\item What do you do with the computer: emails you send, site you visit, documents write...
\item Interact with your hardware: record with the camera/microphone, see the keys you press...
\item Track where you go and which who you are (governments/big organizations don't need special software for this)
\end{itemize}
\end{frame}
\begin{frame}{How to protect yourself}
\begin{itemize}
\item Keep your software updated
\item Use encryption
\item Privilege open source and free software
\item Free = Pay with your privacy
\item Changes on terms of use = Review your privacy settings
\item Two factor authentication
\end{itemize}
\end{frame}
\section{Physical threats}
\begin{frame}{Risk is not just digital}
Would you give me your house keys?\\
Would you let the postman/women enter to  your home?
\end{frame}
\begin{frame}{Communications}
How do you communicate with others?\\
There is a single channel?\\
How do you protect your contacts?
\end{frame}
\begin{frame}{Storage}
How do you store information ?\\
Printed, USB Keys, Hard drives, DVDs...
\end{frame}
\begin{frame}{Protecting your office}
Hide network cables\\
Don't create open networks and change the passwords?\\
\end{frame}
\section{Identifying the assets}
\begin{frame}{Classify the information}
List the information you have:
\begin{itemize}
\item Does it travel?
\item Where is stored
\item How often it should be consulted
\item By how many people
\item Is the content sensitive
\item How long should be kept
\end{itemize}
\end{frame}
\begin{frame}{Sensitive information}
\begin{itemize}
\item Public: everyone will know about it
\item Confidential: only the people of the organization should know about it
\item Secret: only the people involved should know about it
\end{itemize}
\end{frame}
\section{Security indicators: be aware of your environment}
\begin{frame}{What is a security indicator}
Anything situation that makes you aware about a change in security
\end{frame}
\begin{frame}{How to obtain them}
\begin{itemize}
\item Talk with friends, similar groups...
\item Follow the news
\item Meet experts
\end{itemize}
\end{frame}
\begin{frame}{What to do with it}
Share it!  Spread the word and contrast with your colleagues
\end{frame}
\begin{frame}{Tip: the baseline}
How does your usual day look  like?\\
Would you notice if something changed?
\end{frame}
\begin{frame}{What to look for? Digital security indicators}
\begin{itemize}
\item Weird thing on device start
\item Erratic cursor
\item Unusual or unattended emails, texts...
\item Unusual person contact (also on Social networks)
\item Something asking for  too  much information
\item The battery runs out too fast
\end{itemize}
\end{frame}
\begin{frame}{I'm not sure if it's an indicator}
Note it down and \textbf{share it}
The 5 W : When? What? Where? Who? Why? 
\end{frame}
\begin{frame}{What to do when you detect a security indicator}
\begin{itemize}
\item Stop using that service / account
\item Change your passwords asap, specially if email accounts
\end{itemize}
\end{frame}
\section{We identified the threat, now what? protection plan}
\begin{frame}
\begin{center}
\textbf{Everyone needs a plan}
\end{center}
\end{frame}
\begin{frame}
TIP: Don't forget about your well-being
\end{frame}
\begin{frame}
\begin{center}
\textbf{Prevention}\\
\textbf{Emergency plan}
\end{center}
\end{frame}
\subsection{How does a protection plan look like}
\begin{frame}{Assist the refugees - Part I}
\textbf{Objective:}\\
Assist refugees: feed them, give them collected clothing...\\
\textbf{Threats:}\\
\begin{itemize}
\item Harassment of arrest by the police
\item Confiscation of the material
\item Loss of  data and compromise our contacts as a result
\end{itemize}
\textbf{Prevention actions:}\\
\begin{itemize}
\item Alert colleagues that won't be there
\item List the possible aggressors and how to contact them
\item Communicate with someone after the action once everything is finished
\item If you get testimonies or  someone tells you about a right vulneration get the information on a encrypted device
\item Clear your messages, emails...
\item Any live information will be shared via an encrypted messenger and only with people involved
\end{itemize}
\end{frame}
\begin{frame}{Assist the refugees - Part II}
\textbf{Response resources:}\\
\begin{itemize}
\item Prepare emergency messages
\item Memorize and  write down your lawyers number
\end{itemize}
\textbf{Emergency plan:}\\
\begin{itemize}
\item Notify lawyers when someone is arrested
\item Call allied  organizations without disclosing sensitive information
\item Don't talk to the police
\item Encrypt the device before arrest if possible
\end{itemize}
\textbf{Well being considerations:}\\
\begin{itemize}
\item Eat before the action
\item Rest
\end{itemize}
\end{frame}
\subsection{The support network}
\begin{frame}{You are not alone}
The security starts when you feel safe with your colleagues\\
Explain your security concerns to your colleagues\\
Give them emergency contacts and tell them how would you like them to act on emergencies\\
\end{frame}
\begin{frame}{Support network protocols}
If the involved person has specified a protocol follow it. Everyone is responsible of it's own security.\\
Don't disclose any information that could harm involved persons\\
Use secure  communication channels
\end{frame}

\section{Do I really need to  keep  this?}
\begin{frame}
Saving compromising information is dangerous
\end{frame}
\begin{frame}{Can I delete it?}
There is no delete button on the internet\\
On you device pressing the delete button won't delete  it from the drive
\end{frame}
\begin{frame}{So... what do I do?}
\begin{itemize}
\item Clear your cache regularly
\item Wipe your  disk after deleting sensitive information
\end{itemize}
\end{frame}
\begin{frame}{Tools: Deleting the cache}
Deleting the cache:
\begin{itemize}
\item Windows : CCleaner
\item iPhone: Cache Cleaner, Battery Doctor...
\item Android: 
\item Chrome: \href{https://chrome.google.com/webstore/detail/clear-cache/cppjkneekbjaeellbfkmgnhonkkjfpdn}{Clear cache by Benjamin Bojko}
\item Firefox: \href{https://www.hotcleaner.com/clickclean_firefox.html}{Click and Clean}
\end{itemize}
\end{frame}
\begin{frame}{Tools: Wiping your files}
Wiping your files:
\begin{itemize}
\item Windows : \href{https://eraser.heidi.ie/}{Eraser}
\item Linux: \href{http://www.ubuntugeek.com/tools-to-delete-files-securely-in-ubuntu-linux.html}{Shred} (integrated with file browser)
\item MacOS: \href{http://www.brighthub.com/computing/mac-platform/articles/72748.aspx}{Shredder - Empty thrash securely}
\end{itemize}
\end{frame}

\section{For more information}
\begin{frame}
Destroy sensitive information:
\begin{itemize}
\item https://securityinabox.org/en/guide/destroy-sensitive-information/
\end{itemize}
\end{frame}
\end{document}
